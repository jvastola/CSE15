\documentclass[12pt]{article}
\title{Homework 2}
\author{Johnny Vastola}
\date{\today}

\begin{document}
\maketitle
\section{Knowledge Representation}
\paragraph{1.}
It is not cloudy and it is not raining.

\subparagraph{Solution}
Let $\neg p$ = it is not cloudy, and let $q$ = it is not raining, then $\neg p \lor \neg q$

\paragraph{2.}  I like to eat apples and bananas.
\subparagraph{Solution}
Let $p$ = I like to eat apples, and let $q$ = I like to eat bananas, then $p \land q$

\paragraph{3.} Behind the clouds the sun is shining.
\subparagraph{Solution}
Let $p$ = Behind the clouds the sun is shining, then $p$

\paragraph{4.} If a function is differentiable then the function is continuous.
\subparagraph{Solution}
Let $p$ = If a function is differentiable, and let $q$ = the function is continuous, then $p \rightarrow q$

\paragraph{5.} I will study for the final otherwise I will fail.
\subparagraph{Solution}
Let $p$ = I will study for the final, and let $q$ = I will fail, then $p \rightarrow \neg  q$
\section{Equivalence in Propositional Logic}
\begin{enumerate}
  \item $p \land q$  and  $p \lor \neg q$
  Not equivalent, the first statement requires both to be true while the second only needs one to be true if p = false and q = false then 1 yields false while 2 yields true.
  
  \item $p \lor q$  and $ \neg p \lor \neg q$
  Not equivalent, implying the opposite case. The only time the proposition yields false is when both q and p are false and only true if q and q are true.
  \item $p \rightarrow q$  and $\neg q \rightarrow \neg p$
  Equivalent
\begin{tabular}{|c|c|c|c|c|c|}
\hline
$q$ & $p$ & $ \lnot q$ & $ \lnot p$ & $p \to q$ & $ \lnot q \to  \lnot p$ \\
\hline
0 & 0 & 1 & 1 & 1 & 1 \\
0 & 1 & 1 & 0 & 0 & 0 \\
1 & 0 & 0 & 1 & 1 & 1 \\
1 & 1 & 0 & 0 & 1 & 1 \\
\hline
\end{tabular}
  \item $p \rightarrow q$  and $ \neg p \lor q$
  Equivalent
\begin{tabular}{|c|c|c|c|c|}
\hline
$p$ & $q$ & $ \lnot p$ & $p \to q$ & $ \lnot p \lor q$ \\
\hline
0 & 0 & 1 & 1 & 1 \\
0 & 1 & 1 & 1 & 1 \\
1 & 0 & 0 & 0 & 0 \\
1 & 1 & 0 & 1 & 1 \\
\hline
\end{tabular}
  \item $ \neg (p \land q)$   and $ \neg p \lor \neg q$
  Equivalent
\begin{tabular}{|c|c|c|c|c|c|c|}
\hline
$p$ & $q$ & $ \lnot p$ & $ \lnot q$ & $p \land q$ & $ \lnot p \lor \lnot q$ & $ \lnot (p \land q)$ \\
\hline
0 & 0 & 1 & 1 & 0 & 1 & 1 \\
0 & 1 & 1 & 0 & 0 & 1 & 1 \\
1 & 0 & 0 & 1 & 0 & 1 & 1 \\
1 & 1 & 0 & 0 & 1 & 0 & 0 \\
\hline
\end{tabular}
\end{enumerate}


\end{document}

