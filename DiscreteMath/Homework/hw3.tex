

\documentclass[12pt]{article}

\usepackage[margin = 0.5in]{geometry}

\usepackage{amsmath}
\title{Homework 3}

\author{Johnny Vastola}

\date{\today}

\begin{document}

\maketitle



\textbf{Knights and Knaves}

\begin{enumerate}

\item
One day a traveller was wandering around the island of Knights and Knaves, when he encountered two local inhabitants,P and Q.  The traveller asked: "Is any of you a knave?".P replied:  "At least one of us is a knave".\\

Can you find our what P and Q are? If so, what are they? If not, explain why not, and what other information would you need to know.\\

\textit{P is a knight, Q is a knave}
\item
Later on, the traveller met two other locals, A and B.  He asked whether either of them is a knight. A replied:  "If Bis a knave, then I am a knave too".\\

What are A and B?\\

\textit{A is unknown while B must be a knight}

\end{enumerate}



\textbf{Logical Identities}\\

\indent Simplify the following propositions. Show all steps of your solutions.\\

\begin{enumerate}
\item
$\neg(p \to (q \to p))$ \\
$\neg(\neg p \land (q \to p))$\\
$p \land \neg (q \to p))$\\
$p \land \neg (\neg q \lor p)$\\
$p \land (q \land \neg p)$\\
$p \land \neg p \land q$\\
$p \land \neg p = False $\\
$False \land q = False $\\

\item
$\neg((p \land q) \to (q \lor p))$\\
$\neg(\neg (p \land q) \lor (q \lor p))$\\
$(p \land q) \land \neg (q \lor p)$\\
$(p \land q) \land \neg q \land \neg p$\\
$(p \land \neg p) \land \neg q \land \neg q$\\
$False = False$

\end{enumerate}

\textbf{Logical Equivalences}\\

\indent Determine whether or not the following pairs of propositions are equivalent. Show all steps.\\

\begin{enumerate}



\item

p $\to$ (q $\to$ r) and (p $\land$ q) $\to$ r



\begin{tabular}{|c|c|c|c|c|c|c|}
\hline
$q$ & $r$ & $p$ & $q \to r$ & $p \land q$ & $p \to (q \to r)$ & $(p \land q) \to r$ \\
\hline
F & F & F & T & F & T & T \\
F & F & T & T & F & T & T \\
F & T & F & T & F & T & T \\
F & T & T & T & F & T & T \\
T & F & F & F & F & T & T \\
T & F & T & F & T & F & F \\
T & T & F & T & F & T & T \\
T & T & T & T & T & T & T \\
\hline
\end{tabular}

The final two columns depicting the pair of propositions are equivalent
\item

p $\to$ (q $\to$ r) and (p $\to$ q) $\to$ r



\begin{tabular}{|c|c|c|c|c|c|c|}
\hline
$q$ & $r$ & $p$ & $q \to r$ & $p \to q$ & $p \to (q \to r)$ & $(p \to q) \to r$ \\
\hline
F & F & F & T & T & T & F \\
F & F & T & T & F & T & T \\
F & T & F & T & T & T & T \\
F & T & T & T & F & T & T \\
T & F & F & F & T & T & F \\
T & F & T & F & T & F & F \\
T & T & F & T & T & T & T \\
T & T & T & T & T & T & T \\
\hline
\end{tabular}

The pair of propositions are not equivalent as the final two columns depict
\end{enumerate}

\textbf{Logical Consequence}\\

\indent Determine if the following inferences are valid. Explain why. or why not.\\

\begin{enumerate}



\item

Jimmy is smart\\
\underline{Smart people are rich}\\
Jimmy is rich\\

This is a valid argument because it is impossible for the conclusion to be false if we assume the premises are true.
\item

Islands are surrounded by water\\
\underline{Puerto Rico is surrounded by water}\\
Puerto Rico is an island\\

This is not valid because being an island is a subset of being surrounded by water therefore you cant infer that something (Puerto Rico) is an island because it is surrounded by water
\end{enumerate}

\end{document}